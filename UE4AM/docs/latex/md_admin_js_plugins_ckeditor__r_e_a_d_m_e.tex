Copyright (c) 2003-\/2013, C\-K\-Source -\/ Frederico Knabben. All rights reserved. \href{http://ckeditor.com}{\tt http\-://ckeditor.\-com} -\/ See L\-I\-C\-E\-N\-S\-E.\-md for license information.

C\-K\-Editor is a text editor to be used inside web pages. It's not a replacement for desktop text editors like Word or Open\-Office, but a component to be used as part of web applications and websites.

\subsection*{Documentation}

The full editor documentation is available online at the following address\-: \href{http://docs.ckeditor.com}{\tt http\-://docs.\-ckeditor.\-com}

\subsection*{Installation}

Installing C\-K\-Editor is an easy task. Just follow these simple steps\-:
\begin{DoxyEnumerate}
\item {\bfseries Download} the latest version from the C\-K\-Editor website\-: \href{http://ckeditor.com}{\tt http\-://ckeditor.\-com}. You should have already completed this step, but be sure you have the very latest version.
\item {\bfseries Extract} (decompress) the downloaded file into the root of your website.
\end{DoxyEnumerate}

{\bfseries Note\-:} C\-K\-Editor is by default installed in the {\ttfamily ckeditor} folder. You can place the files in whichever you want though.

\subsection*{Checking Your Installation}

The editor comes with a few sample pages that can be used to verify that installation proceeded properly. Take a look at the {\ttfamily samples} directory.

To test your installation, just call the following page at your website\-: \begin{DoxyVerb}http://<your site>/<CKEditor installation path>/samples/index.html
\end{DoxyVerb}


For example\-: \begin{DoxyVerb}http://www.example.com/ckeditor/samples/index.html\end{DoxyVerb}
 